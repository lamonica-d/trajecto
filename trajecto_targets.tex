% Options for packages loaded elsewhere
\PassOptionsToPackage{unicode}{hyperref}
\PassOptionsToPackage{hyphens}{url}
\PassOptionsToPackage{dvipsnames,svgnames,x11names}{xcolor}
%
\documentclass[
  letterpaper,
  DIV=11,
  numbers=noendperiod]{scrartcl}

\usepackage{amsmath,amssymb}
\usepackage{iftex}
\ifPDFTeX
  \usepackage[T1]{fontenc}
  \usepackage[utf8]{inputenc}
  \usepackage{textcomp} % provide euro and other symbols
\else % if luatex or xetex
  \usepackage{unicode-math}
  \defaultfontfeatures{Scale=MatchLowercase}
  \defaultfontfeatures[\rmfamily]{Ligatures=TeX,Scale=1}
\fi
\usepackage{lmodern}
\ifPDFTeX\else  
    % xetex/luatex font selection
\fi
% Use upquote if available, for straight quotes in verbatim environments
\IfFileExists{upquote.sty}{\usepackage{upquote}}{}
\IfFileExists{microtype.sty}{% use microtype if available
  \usepackage[]{microtype}
  \UseMicrotypeSet[protrusion]{basicmath} % disable protrusion for tt fonts
}{}
\makeatletter
\@ifundefined{KOMAClassName}{% if non-KOMA class
  \IfFileExists{parskip.sty}{%
    \usepackage{parskip}
  }{% else
    \setlength{\parindent}{0pt}
    \setlength{\parskip}{6pt plus 2pt minus 1pt}}
}{% if KOMA class
  \KOMAoptions{parskip=half}}
\makeatother
\usepackage{xcolor}
\setlength{\emergencystretch}{3em} % prevent overfull lines
\setcounter{secnumdepth}{-\maxdimen} % remove section numbering
% Make \paragraph and \subparagraph free-standing
\ifx\paragraph\undefined\else
  \let\oldparagraph\paragraph
  \renewcommand{\paragraph}[1]{\oldparagraph{#1}\mbox{}}
\fi
\ifx\subparagraph\undefined\else
  \let\oldsubparagraph\subparagraph
  \renewcommand{\subparagraph}[1]{\oldsubparagraph{#1}\mbox{}}
\fi

\usepackage{color}
\usepackage{fancyvrb}
\newcommand{\VerbBar}{|}
\newcommand{\VERB}{\Verb[commandchars=\\\{\}]}
\DefineVerbatimEnvironment{Highlighting}{Verbatim}{commandchars=\\\{\}}
% Add ',fontsize=\small' for more characters per line
\usepackage{framed}
\definecolor{shadecolor}{RGB}{241,243,245}
\newenvironment{Shaded}{\begin{snugshade}}{\end{snugshade}}
\newcommand{\AlertTok}[1]{\textcolor[rgb]{0.68,0.00,0.00}{#1}}
\newcommand{\AnnotationTok}[1]{\textcolor[rgb]{0.37,0.37,0.37}{#1}}
\newcommand{\AttributeTok}[1]{\textcolor[rgb]{0.40,0.45,0.13}{#1}}
\newcommand{\BaseNTok}[1]{\textcolor[rgb]{0.68,0.00,0.00}{#1}}
\newcommand{\BuiltInTok}[1]{\textcolor[rgb]{0.00,0.23,0.31}{#1}}
\newcommand{\CharTok}[1]{\textcolor[rgb]{0.13,0.47,0.30}{#1}}
\newcommand{\CommentTok}[1]{\textcolor[rgb]{0.37,0.37,0.37}{#1}}
\newcommand{\CommentVarTok}[1]{\textcolor[rgb]{0.37,0.37,0.37}{\textit{#1}}}
\newcommand{\ConstantTok}[1]{\textcolor[rgb]{0.56,0.35,0.01}{#1}}
\newcommand{\ControlFlowTok}[1]{\textcolor[rgb]{0.00,0.23,0.31}{#1}}
\newcommand{\DataTypeTok}[1]{\textcolor[rgb]{0.68,0.00,0.00}{#1}}
\newcommand{\DecValTok}[1]{\textcolor[rgb]{0.68,0.00,0.00}{#1}}
\newcommand{\DocumentationTok}[1]{\textcolor[rgb]{0.37,0.37,0.37}{\textit{#1}}}
\newcommand{\ErrorTok}[1]{\textcolor[rgb]{0.68,0.00,0.00}{#1}}
\newcommand{\ExtensionTok}[1]{\textcolor[rgb]{0.00,0.23,0.31}{#1}}
\newcommand{\FloatTok}[1]{\textcolor[rgb]{0.68,0.00,0.00}{#1}}
\newcommand{\FunctionTok}[1]{\textcolor[rgb]{0.28,0.35,0.67}{#1}}
\newcommand{\ImportTok}[1]{\textcolor[rgb]{0.00,0.46,0.62}{#1}}
\newcommand{\InformationTok}[1]{\textcolor[rgb]{0.37,0.37,0.37}{#1}}
\newcommand{\KeywordTok}[1]{\textcolor[rgb]{0.00,0.23,0.31}{#1}}
\newcommand{\NormalTok}[1]{\textcolor[rgb]{0.00,0.23,0.31}{#1}}
\newcommand{\OperatorTok}[1]{\textcolor[rgb]{0.37,0.37,0.37}{#1}}
\newcommand{\OtherTok}[1]{\textcolor[rgb]{0.00,0.23,0.31}{#1}}
\newcommand{\PreprocessorTok}[1]{\textcolor[rgb]{0.68,0.00,0.00}{#1}}
\newcommand{\RegionMarkerTok}[1]{\textcolor[rgb]{0.00,0.23,0.31}{#1}}
\newcommand{\SpecialCharTok}[1]{\textcolor[rgb]{0.37,0.37,0.37}{#1}}
\newcommand{\SpecialStringTok}[1]{\textcolor[rgb]{0.13,0.47,0.30}{#1}}
\newcommand{\StringTok}[1]{\textcolor[rgb]{0.13,0.47,0.30}{#1}}
\newcommand{\VariableTok}[1]{\textcolor[rgb]{0.07,0.07,0.07}{#1}}
\newcommand{\VerbatimStringTok}[1]{\textcolor[rgb]{0.13,0.47,0.30}{#1}}
\newcommand{\WarningTok}[1]{\textcolor[rgb]{0.37,0.37,0.37}{\textit{#1}}}

\providecommand{\tightlist}{%
  \setlength{\itemsep}{0pt}\setlength{\parskip}{0pt}}\usepackage{longtable,booktabs,array}
\usepackage{calc} % for calculating minipage widths
% Correct order of tables after \paragraph or \subparagraph
\usepackage{etoolbox}
\makeatletter
\patchcmd\longtable{\par}{\if@noskipsec\mbox{}\fi\par}{}{}
\makeatother
% Allow footnotes in longtable head/foot
\IfFileExists{footnotehyper.sty}{\usepackage{footnotehyper}}{\usepackage{footnote}}
\makesavenoteenv{longtable}
\usepackage{graphicx}
\makeatletter
\def\maxwidth{\ifdim\Gin@nat@width>\linewidth\linewidth\else\Gin@nat@width\fi}
\def\maxheight{\ifdim\Gin@nat@height>\textheight\textheight\else\Gin@nat@height\fi}
\makeatother
% Scale images if necessary, so that they will not overflow the page
% margins by default, and it is still possible to overwrite the defaults
% using explicit options in \includegraphics[width, height, ...]{}
\setkeys{Gin}{width=\maxwidth,height=\maxheight,keepaspectratio}
% Set default figure placement to htbp
\makeatletter
\def\fps@figure{htbp}
\makeatother

\KOMAoption{captions}{tableheading}
\makeatletter
\makeatother
\makeatletter
\makeatother
\makeatletter
\@ifpackageloaded{caption}{}{\usepackage{caption}}
\AtBeginDocument{%
\ifdefined\contentsname
  \renewcommand*\contentsname{Table of contents}
\else
  \newcommand\contentsname{Table of contents}
\fi
\ifdefined\listfigurename
  \renewcommand*\listfigurename{List of Figures}
\else
  \newcommand\listfigurename{List of Figures}
\fi
\ifdefined\listtablename
  \renewcommand*\listtablename{List of Tables}
\else
  \newcommand\listtablename{List of Tables}
\fi
\ifdefined\figurename
  \renewcommand*\figurename{Figure}
\else
  \newcommand\figurename{Figure}
\fi
\ifdefined\tablename
  \renewcommand*\tablename{Table}
\else
  \newcommand\tablename{Table}
\fi
}
\@ifpackageloaded{float}{}{\usepackage{float}}
\floatstyle{ruled}
\@ifundefined{c@chapter}{\newfloat{codelisting}{h}{lop}}{\newfloat{codelisting}{h}{lop}[chapter]}
\floatname{codelisting}{Listing}
\newcommand*\listoflistings{\listof{codelisting}{List of Listings}}
\makeatother
\makeatletter
\@ifpackageloaded{caption}{}{\usepackage{caption}}
\@ifpackageloaded{subcaption}{}{\usepackage{subcaption}}
\makeatother
\makeatletter
\@ifpackageloaded{tcolorbox}{}{\usepackage[skins,breakable]{tcolorbox}}
\makeatother
\makeatletter
\@ifundefined{shadecolor}{\definecolor{shadecolor}{rgb}{.97, .97, .97}}
\makeatother
\makeatletter
\makeatother
\makeatletter
\makeatother
\ifLuaTeX
  \usepackage{selnolig}  % disable illegal ligatures
\fi
\IfFileExists{bookmark.sty}{\usepackage{bookmark}}{\usepackage{hyperref}}
\IfFileExists{xurl.sty}{\usepackage{xurl}}{} % add URL line breaks if available
\urlstyle{same} % disable monospaced font for URLs
\hypersetup{
  pdftitle={Classification of fish behavioral responses to environmental fluctuations in a large river},
  pdfauthor={Dominique Lamonica1,2; Hilaire Drouineau3; Hervé Capra1; Hervé Pella1; Jérémy Piffady1; Anthony Maire4},
  colorlinks=true,
  linkcolor={blue},
  filecolor={Maroon},
  citecolor={Blue},
  urlcolor={Blue},
  pdfcreator={LaTeX via pandoc}}

\title{Classification of fish behavioral responses to environmental
fluctuations in a large river}
\author{Dominique Lamonica\textsuperscript{1,2} \and Hilaire
Drouineau\textsuperscript{3} \and Hervé
Capra\textsuperscript{1} \and Hervé Pella\textsuperscript{1} \and Jérémy
Piffady\textsuperscript{1} \and Anthony Maire\textsuperscript{4}}
\date{}

\begin{document}
\maketitle
\ifdefined\Shaded\renewenvironment{Shaded}{\begin{tcolorbox}[borderline west={3pt}{0pt}{shadecolor}, interior hidden, boxrule=0pt, frame hidden, enhanced, breakable, sharp corners]}{\end{tcolorbox}}\fi

1 - INRAE, RiverLy, HYNES (Irstea-EDF R\&D), Villeurbanne, F-69625,
France\\
2 - IRD\\
3 - INRAE, EABX, HYNES (Irstea-EDF R\&D), Cestas, F-33612, France\\
4 - EDF R\&D, LNHE (Laboratoire National d'Hydraulique et
Environnement), HYNES (Irstea - EDF R\&D), 6 quai Watier, 78401 Chatou
Cedex, France\\
\textbf{Corresponding author:} Dominique Lamonica, address: , email:
dominique.lamonica@ird.fr\\
\textbf{Running title:}\\
\textbf{Word count:} words

\textbf{Keywords:}

\hypertarget{description}{%
\section{Description}\label{description}}

This quarto file was produced to present the trajecto project. This file
is supposed to be integrated in the target pipeline.

\hypertarget{packages}{%
\subsection{Packages}\label{packages}}

These are the packages I used for my file \textbf{test\_targets.R} file.

\begin{Shaded}
\begin{Highlighting}[]
\CommentTok{\# library(targets)}
\CommentTok{\# library(tarchetypes)}
\CommentTok{\# library(ggplot2)}
\CommentTok{\# library(visNetwork)}
\end{Highlighting}
\end{Shaded}

These are the packages I might use this file \emph{.qmd}.

\hypertarget{setup}{%
\subsection{Setup}\label{setup}}

These are the parameters to apply to this whole document
(\textbf{report\_targets.qmd}).

\hypertarget{introduction}{%
\section{Introduction}\label{introduction}}

Movement is a key feature of living organisms, affecting many ecological
processes. In the last decades, the development of individual marking
and tracking methods and related analytical methods have revolutionised
behavioural ecology , especially movement ecology, which studies the
link between the environment, organism internal states and their
resulting movement. Most movements relate to the need for resources such
as food or mates . Station keeping refer to small scale movements within
the home range of individuals. It includes forging, i.e.~a small scale
and generally regular movement to find appropriate resource. It also
included the defence of the habitat or predation avoidance and do not
require complex orientation and navigation systems {[}@dingle2006{]}. On
the other hand, ranging refers to larger scale exploratory movements
outside the home range to find more suitable habitats, that stop when
the appropriate resources are found . Lastly, migration refer to an
oriented compulsory movement outside the home range, occurring on a
seasonal basis with a return movement . Analysing individual movements
can inform on their behaviours and on their habitats needs. More
importantly, movements can be used to assess the perturbations due to
anthropogenic modifications of habitats.

Rivers provide multiple ecosystem services, but are heavily anthropised
and modified by the implementation of dams or weirs, by water
intake/outake, regulation of discharges, modification of habitats ... .
At large temporal and spatial scales, Arevalo et al. show how global
change have modified the thermal and discharge regimes of rivers, and
how such modifications can in turn impact habitats suitability by fish .
At a smaller spatial scale, the modifications of water velocity or
thermal regimes due to anthropogenic pressures also modify the spatial
distribution of fishes permanently or temporarily . At both scales, both
intraspecific and interspecific variabilities are observed. High
resolution trajectories data appears very relevant to improve our
understanding of the responses of fishes to natural of human-induced
variations in environmental conditions. Many dedicated methods and
models have been developed to analysis such data. They would allow
quantifying the changes behaviours and habitat use and explore for the
existence of patterns of responses.

Yet, contrary to lakes or marine ecosystems, there is only a few studies
investigating the influence on local environment in river on fish
through the analysis of their trajectories. Many of these analysis focus
on fish migration to assess the impacts of dams and weirs . Indeed,
specific difficulties of movement data acquisition/defects of available
data and data analysis are encountered in rivers. Most positioning
systems use a triangulation procedure which is highly sensitive to
regularities in banks profiles and ground irregularities . As such, the
precision is spatially very variable . Moreover, contrary to open
systems, rivers are highly anisotropic with banks and flow direction
constraining fish movements . Given all these specificities, Lamonica et
al.~developed a specific framework to preprocess fish trajectories data
aiming to address these limitations and then, to facilitate the
exploration of the impacts of environmental conditions on fish
behaviours.

Behavioural ecology has shown intraspecies variability in habitat
preference and use (refs). This compels several processes of the
population dynamics (dev un peu). Similarly, variations in species
response to environmental changes can impact the community diversity,
distribution and abundance (refs?). To investigate individual behaviour
response to local environment variations, movement data appear to be an
appropriate window. Several statistical and modelling tools are
available to analyse and link movement data to individual behaviour
(Gurarie2016). In particular, state-space models are widely used, and
they offer the possibility of integrating environmental influence on
switching probabilities between behavioural states (dev examples/refs).

In this paper, we propose to assess: 1) the effect of local
environmental factors on behaviour change of individuals among
freshwater fish species, and 2) the inter- and intraspecies variability
of the response to environmental variations. We chose three freshwater
species (chubs, barbel and catfish) and we focused on three major
environmental factors, namely water depth, flow velocity and water
temperature (why those factors + ref plichard2017). Our hypotheses are
the following:.\\
We developed a movement state-space model describing the effect of flow
velocity, water depth and water temperature on behavioural change
probabilities to investigate the variations in the effect of local
environment on individual habitat use of freshwater fish (Figure 1).
Movement data was derived from observed telemetry data, \emph{i.e.}
location and time, and local environment input data was simulated using
a 2D hydraulic model. For each individual, independently model
parameters were inferred within a Bayesian framework. Behavioural change
probability functions were computed with the posterior distributions of
parameters describing the effect of environment on behavioural change.
For each environmental variable, independent classifications were then
performed to classify individuals based on probability function shapes
which describe their responses to a change in this environmental
variable.\\

\hypertarget{materials-and-methods}{%
\section{Materials and Methods}\label{materials-and-methods}}

\hypertarget{study-area}{%
\subsection{Study area}\label{study-area}}

The study river section is located on the Rhône river and is 1.8km long
and 140m wide (at a mean discharge of 465m\textsuperscript{3}
s\textsuperscript{-1}). This river section is situated 363km upstream
from the river mouth, near the Bugey nuclear power plant 45°47'44''N;
5°16'25''E. The power plant abstracts c.a. 100m\textsuperscript{3}
s\textsuperscript{-1} at the upstream end of our study reach to cool its
four reactors, and releases warmed water (between 7°C and 10°C warmer
than the upstream water) at two different locations , creating a
temperature difference between the left and the right bank. More details
on the study site can be found in Plichard2017.

\hypertarget{data-collection-and-simulation}{%
\subsection{Data collection and
simulation}\label{data-collection-and-simulation}}

\hypertarget{movementtrajectory-data}{%
\subsubsection{Movement/trajectory data}\label{movementtrajectory-data}}

Bergé et al.~collected telemetry data on 94 freshwater fish individuals
of X species in the Rhône river using an HTI
(\url{https://www.innovasea.com/fish-tracking/}) acoustic fixed
telemetry system with X hydrophones positioned on the riverbed
{[}@Berge2012{]}. Locations of these individuals have been tracked
during three months at a period of three seconds. Each tag emits a
specific acoustic signal/at a specific frequency that allows the
identification of fish individual and precise positioning through a
triangulation process, provided that the signal is detected by at least
3 hydrophones. Those data have been pre-treated according to the method
presented in Lamonica2020 to handle usual defects of telemetry data
(small and large gaps in location recording, and specific artefactual
patterns in some trajectories due to the triangulation process of the
HTI system). For each individual, we obtained a set of independent
trajectories with locations regularly spaced in time (1 minute). We
selected the individuals which have been located more than xx hours
(number of locations), namely 4 individuals of barbel \emph{Barbus
barbus}, 8 individuals of European chub \emph{Squalius cephalus} and 6
individuals of European catfish \emph{Silurus glanis}.

\hypertarget{data-acquisition}{%
\subsubsection{Data acquisition}\label{data-acquisition}}

Show the data frame of clean data for each individuals

\begin{Shaded}
\begin{Highlighting}[]
\FunctionTok{tar\_load}\NormalTok{(data\_for\_inference)}
\end{Highlighting}
\end{Shaded}

\begin{verbatim}
✖ Identified no targets to load.
\end{verbatim}

\hypertarget{environment-data}{%
\subsubsection{Environment data}\label{environment-data}}

A two-dimensional unsteady hydraulic model of the reach considering the
longitudinal and lateral variations of depth-averaged velocities and
accounting for discharge variations have been used to simulate fish
local environment (Telemac 2D model, calibration details in Capra et
al., 2011). The Telemac 2D model also had a component that allowed us to
simulate water temperature conditions across the reach (Hervouet, 1999)
. For each time and location of each individual, depth-averaged velocity
(\(V\), m.s\textsuperscript{-1}), water depth (\(D\), m) and the
difference between the local water temperature and the water temperature
recorded upstream of the study section (\(T\), \(^{\circ}\)C) have been
simulated. Then those three variables have been normalised across all
individuals and locations in order to compare the effect of the
variables between individuals and between the variables themselves on
fish movement/behaviour.

\hypertarget{state-space-model-parameter-inference}{%
\subsection{State space model + parameter
inference}\label{state-space-model-parameter-inference}}

\hypertarget{state-space-modelling}{%
\subsubsection{State-space modelling}\label{state-space-modelling}}

We developed a state-space model based on Lamonica2020 aiming at
discriminating the different individual behaviours displayed by an
individual during its trajectories, and at describing how fishes switch
between those behaviours as a response to environmental cues. Model
states correspond to the succession of fish behaviours at each
time-step, with two possible behaviours: ``Resting'' (denoted by \(R\))
which corresponds to slow or erratic short-distance movements, and
``Moving'' (denoted by \(M\)) which corresponds to fast oriented
movements (Lamonica2020). We assumed a quadratic influence of each
environmental variable on the behaviour switching probabilities between
each time-step, with a logit link to map the quadratic linear function
to switching probabilities . The quadratic function allows for
non-linear relationships, especially dome-shape relationships, between
environmental factors and behaviour changes. We also added a fixed
effect of the nychthemeral period (day, dust, dawn, night) to account
for possible different levels of activity throughout the day. The
observation model links the state at time \(t\) to corresponding
movement variables (\emph{i.e.} speed between two locations and turning
angles between two moves) and two additional variables which summarize
the raw data to help discriminating between behaviours Lamonica2020. The
model is written as follows:\\
Transition matrix \[M_{q}(t)=
\begin{pmatrix}
   1-q_{R \rightarrow M}(t) &q_{R \rightarrow M}(t) \\
   q_{M \rightarrow R}(t)&1-q_{M \rightarrow R}(t) 
\end{pmatrix}\]

\[\text{logit}(q_{R \rightarrow M}(t)) = M_{a} \times M_{E}(t)\]

\[\text{logit}(q_{M \rightarrow R}(t)) = M_{b} \times M_{E}(t)\]

with \[M_{a}=(a_{1}, ..., a_{11})\]

\[M_{b}=(b_{1}, ..., b_{11})\] and

\[M_{E}(t)=
\begin{pmatrix}
   0 \\
   V(t) \\
   V^2(t) \\
   D(t) \\
   D^2(t) \\
   T(t) \\
   T^2(t) \\
   \mathbb{1}_{DAY}(t) \\
   \mathbb{1}_{DUST}(t) \\
   \mathbb{1}_{NIGHT}(t) \\
   \mathbb{1}_{DAWN}(t)
\end{pmatrix}\]

State equation \[z_{t} \sim \mathcal{B}ernoulli(M_{q}(t)[z_{t-1}])\]
Observation model

\[\begin{array}{l}
y_{v_{t}} \sim \mathcal{G}amma(k[z_{t}], \theta[z_{t}])\\
y_{\phi_{t}} \sim \mathcal{V}on\mathcal{M}ises(m[z_{t}], \rho[z_{t}])\\
U_{1_{t}} \sim \mathcal{B}eta(\alpha[z_{t}], \beta[z_{t}])\\
U_{2_{t}} \sim \mathcal{N}ormal(\mu[z_{t}], \sigma[z_{t}])
\end{array}\]

with \(q_{R \rightarrow M}\) and \(q_{M \rightarrow R}\) the probability
of switching from resting, respectively moving, behaviour to moving,
respectively resting, behaviour, \(z_{t}\) the behaviour at time \(t\)
(\(R\) or \(M\)). Parameters \(a_{1}\) and \(b_{1}\) are the intercepts
of the switching probabilities, \(a_{2}\) and \(b_{2}\), and \(a_{5}\)
and \(b_{5}\) are the linear and quadratic coefficients related to flow
velocity, \(a_{3}\) and \(b_{3}\), and \(a_{6}\) and \(b_{6}\) are the
linear and quadratic coefficients related to water depth, \(a_{4}\) and
\(b_{4}\), and \(a_{7}\) and \(b_{7}\) are the linear and quadratic
coefficients related to temperature difference.\\
Concerning the data, \(y_{v_{t}}\) is the observed speed between \(t-1\)
and \(t\) and \(y_{\phi_{t}}\) is the observed turning angles between
\(t-2\) and \(t-1\) and \(t-1\) and \(t\), \(U_{1_{t}}\) is the ratio
between the covered distance from interpolated data and the covered
distance from raw data between \(t-1\) and \(t\), and \(U_{2_{t}}\) is
the variance of turning angles from raw data between \(t-1\) and \(t\).
Concerning parameters, \(k[z_{t}]\) and \(\theta[z_{t}]\) are the
parameters describing the speed for behaviour \(z\) at time \(t\) and
\(m[z_{t}]\) and \(\rho[z_{t}]\) are the parameters describing the
turning angle for behaviour \(z\) at time \(t\), with \(\alpha[z_{t}]\)
and \(\beta[z_{t}]\) the parameters describing \(U_{1}\) for behaviour
\(z\) at time \(t\), \(\mu[z_{t}]\) and \(\sigma[z_{t}]\) the parameters
describing \(U_{2}\) for behaviour \(z\) at time \(t\).

\hypertarget{parameter-inference}{%
\subsubsection{Parameter inference}\label{parameter-inference}}

Bayesian inference was used to fit the model to the data independently
for each individual. We defined prior distributions summarising all
available information on each parameter (see SI Table xx). Markov Chain
Monte Carlo (MCMC) computations were performed using JAGS software and
the \emph{rjags} R package. A total of 5,000 iterations were performed
as a burn-in phase and inference was based on 20,000 to 70,000
additional iterations, depending on the individual, for each of the
three independent chains. We used the Gelman and Rubin tests to check
the convergence of the estimation process.

\hypertarget{clustering-of-fishes}{%
\subsection{Clustering of fishes}\label{clustering-of-fishes}}

We aimed at classifying the response of individuals to environmental
factors in terms of behavioral change, \emph{i.e.} based on how the
probability of behaviour change varies along the environmental gradient.
For that we used the derivative of the probability functions according
to the target environmental variable, with the two other variables fixed
to zero (\emph{i.e.} at their average value). The general form of the
derivative function for the environmental variable \(X\) is thus written
as follow:
\[\frac{(c + 2\times d\times X)\exp(a + b + c\times X + d\times X^2)}{\exp(a + b + c\times X + d\times X^2) + 1} -
\frac{(c + 2\times d\times X)\exp(2\times (a + b) + 2\times c\times X + 2\times d\times X^2)}{(\exp(a + b + c\times X + d\times X^2) + 1)^2}\]
with \(a\) the intercept of the considered switching probability (from
Resting to Moving \(q_{R \rightarrow M}\), or Moving to Resting
\(q_{M \rightarrow R}\)), \(b\) the coefficient of the nychthemeral
period fixed effect, \(c\) and \(d\) being the linear and quadratic
coefficients respectively, relative to the considered environmental
variable \(X\) (flow velocity \(V\), water depth \(D\), or water
temperature difference \(T\)).\\
Six independent classifications were performed, \emph{i.e.} two
behaviour changing probabilities X three environmental variables. To
carry out a clustering analysis, a metric is required to quantify the
distance/similarity between each pair of functions. In this aim, we
calculated for each pair of fucntions the absolute area between the two
curves, restrained on the main part of the environmental gradient
(x-axis range of standardised environmental variable = \[-1.5; 1.5\]),
using area between curve function from the \emph{geiger} R package
(ref).\\
In order to take into account the uncertainties in estimates of the
transition matrix coefficients, we performed the classification using
\(n\) replicates for each individual. Those replicates were generated by
randomly selecting \(n\) iterations of the posterior distributions,
leading to \(n\) values of the transition matrix coefficients which have
been used to compute \(n\) derivative functions per individual,
according to equation (6).\\
For each of the 6 independent classifications, we performed hierarchical
clustering using whatever function from whatever R package (ref). We set
the number of classes according to inertia decrease (ref?) (see SI
figure Ax). Each replicate of each individual fell into a class. To
assign an individual to a class, we selected the one where most of the
replicates fell into.

\hypertarget{results}{%
\section{Results}\label{results}}

\hypertarget{nychthemeral-period-effect}{%
\subsection{Nychthemeral period
effect}\label{nychthemeral-period-effect}}

For all individuals and for both switching behaviour probabilities, at
least the coefficient of the fixed effect for one nychthemeral period
was significantly different from the others (ANOVA SI Table 2, Figure
S2). However, the coefficients of nychthemeral period fixed effects
showed no general pattern. We chose to focus the results on the
switching probabilities computed for the day.

\hypertarget{general-effect-of-the-environment}{%
\subsection{General effect of the
environment}\label{general-effect-of-the-environment}}

Outside of the extreme range of environmental values, median switching
probabilities were below 0.5 (SI Figure S4), suggesting that individuals
tend to stay in the same behaviour. Overall, the effect of the water
temperature difference on switching probabilities was weaker than the
effect of water depth and flow velocity, as suggested by the boxplots of
the linear and quadratic coefficients being closer to zero for
temperature difference (SI Figure S3). Hence, the median switching
probabilities were less variable, almost constant, with temperature
difference for all individuals (SI Figure S4). Consequently, we
performed the classification on switching probabilities functions only
for water depth and flow velocity. For those two environmental factors,
there was no general pattern: probability functions were found to be
monotonously increasing or decreasing, or bell shaped concave or convex
(SI Figure S4). Simulated responses to water depth and flow velocity
(with temperature difference equal to 0) for the value ranges
encountered by each individual are showed Figure 2. Overall, in the
extreme values of the environmental gradient, the probability of
behaviour switch/change increases, in particular from resting to moving.
However, for some individuals one or both switching probabilities showed
very little variations according to the environmental factors.

\hypertarget{inter--and-intra-species-variability}{%
\subsection{Inter- and intra-species
variability}\label{inter--and-intra-species-variability}}

For the probability to switch from moving to resting according to flow
velocity, three groups were identified (Figure 3a): the individuals from
group one (red) showed an increasing then slightly decreasing
probability, the individuals from the second group (green) showed the
inverse pattern, while for the third group (blue) the probability showed
little variations. For the probability to switch from resting to moving
according to flow velocity, four groups were identified (Figure 3b): the
individuals from group one (red) showed a concave parabolic probability,
the third group (blue) showed the inverse pattern, the individuals from
the second group (green) showed a decreasing probability. For the
probability to switch from moving to resting according to water depth,
two groups were identified (Figure 3c): the individuals from group one
(red) showed a concave parabolic probability, the individuals from the
second group (blue) showed a convex parabolic probability or less
variations. For the probability to switch from resting to moving
according to water depth, three groups were identified (Figure 3d): the
individuals from group one (red) showed a concave parabolic probability,
the third group (blue) showed the inverse pattern, while for the second
group (blue) the probability was overall steadily increasing.\\
Overall, classification of switching probability functions according to
water depth or flow velocity did not allow to identify any species- or
size- specific patterns (Figure 3). In addition, the groups were not
consistent across the couples probability-environmental factor,
suggesting that there is no pattern in the global response to the
environmental conditions.

\hypertarget{discussion}{%
\section{Discussion}\label{discussion}}

\begin{itemize}
\item
  what do those probability transitions mean ecologically, for the fish
  behaviour ?
\item
  how can we interpret the effect of environment according to habitat
  preferences and what we know about those species ecology ?
\item
  such an individual variability that we can not see any species/size
  patterns, given the limitations of the experiment. develop on exp
  limitations (river, telemetry, size of the section, loss of the
  signal, etc.)
\end{itemize}

\hypertarget{conclusions}{%
\section{Conclusions}\label{conclusions}}

\begin{itemize}
\item
  summary of results/interpretations: minor effect of water temperature
  difference compared to flow velocity and water depth + no general
  pattern in the local environment effect, strong inter individual
  variability + no group pattern + it seems that individual preferences,
  or/and other local environmental factors, are the major driver of
  behaviour changes.
\item
  perspectives: the model can help inferring/predicting in which
  behavioral state is/will be an individual, as well as changes in
  behaviour. In an environment with such a high anthropic pressure,
  knowing those behaviour modifications is helpful to assess how
  disturbed the individuals can be.
\end{itemize}

\hypertarget{figures}{%
\section{Figures}\label{figures}}



\end{document}
